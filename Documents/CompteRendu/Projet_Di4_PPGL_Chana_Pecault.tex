%fichier : Projet_Di4_PPGL_Chana_Pecault.tex
%Date : 21/09/2021
%Version : 1.00
%Modif : 21/09/2021

\documentclass{EPUProjetDi}

\makeindex

%remplir les lignes suivantes avec les informations vous concernant :
\title[Simulation d'une colonie de fourmis]{Simulation d'une colonie de fourmis}

\projet{Projet de Programmation et Génie Logiciel}

\author{
Narvin Chana\\ %Attention : toujours écrire d'abord le prénom puis le nom (ne pas mettre tout le nom en majuscules)
\noindent[\url{narvin.chana@etu.univ-tours.fr}]\\
Noé Pécault\\ %Attention : toujours écrire d'abord le prénom puis le nom (ne pas mettre tout le nom en majuscules)
\noindent[\url{noe.pecault@etu.univ-tours.fr}]}

\encadrant{Nicolas Monmarché\\ %
\noindent[\url{nicolas.monmarche@univ-tours.fr}]\\~\\
Polytech Tours\\
Département Informatique\\~ %
}

%%%%%%%%%%%%%%%%%%%%%%%%%%%%%%%%%%%%%%%%%%%%%%%%%%%%%%%%%%%%%%%%%%%%%%%%%%%%%%%%%%%%%%%%%%
\begin{document}

\maketitle

\pagenumbering{roman}
\setcounter{page}{0}

{
%on réduit momentanément l'écart entre paragraphes pour ne pas trop aérer la table des matières
\setlength{\parskip}{0em}

\tableofcontents

%\listoffigures
%rq1 : si vous n'avez pratiquement pas de figures, laissez la ligne précédente en commentaire

%\listoftables
%rq1 : si vous n'avez pratiquement pas de tables, laissez la ligne précédente en commentaire
}


\start
%%%%%%%%%%%%%%%%%%%%%%%%%%%%%%%%%%%%%%%%%%%%%%%%%%%%%%%%%%%%%%%%%%%%%%%%%%%%%%%%%%%%%%%%%%

\chapter*{Introduction}
%le 2 lignes suivantes permettent d'ajouter l'introduction à la table des matières
%et d'afficher "Introduction en haut des pages"
\addcontentsline{toc}{chapter}{\numberline{}Introduction}
\markboth{\hspace{0.5cm}Introduction}{}

\paragraph{}
Le problème de la simulation et de l'optimisation d'une colonie de fourmis est un problème complexe 
qui met en oeuvre des connaissances dans les domaines de la théorie des graphes, la recherche opérationelle 
et dans les systèmes décentralisés. Chaque fourmi agi par elle même et l'ensemble des fourmis forme une intelligence
à part entière qui n'a pas de cerveau centralisé.

Ses applications sont vastes et interviennent dans des domaines tels que les problèmes d'ordonnancements, 
de routage (ex: réseau Internet ou tournée de véhicules) ou encore le traitement d'image (ex: détection de bords).

Ce problème de simulation de colonie de fourmis nous intéresse depuis longtemps et c'est donc pour cela que dans 
le cadre du projet de programmation et génie logiciel nous avons choisi de proposer un sujet qui traite de celui-ci.

Notre objectif pour ce projet était de créer une application permettant de simuler les interactions qu'une colonie 
de fourmis peut avoir avec son environnement. Les fourmis doivent pouvoir explorer les alentours de leur colonie afin 
de trouver de la nourriture et de la ramener à sa colonie. 
Nous avions aussi imaginé d'autres fonctionnalités qui pourraient venir se greffer au projet si le temps nous le permettait
comme des combats entre fourmilières et la possibilité de simuler plusieurs colonies.

En revanche, il n'était pas dans nos objectifs de simuler l'intérieur de la colonie et n'est pas voué à être sur le long terme
(nous ne prenions donc pas en compte la durée de vie des fourmis).

Il nous était également très important de respecter les principes de génie logiciel qui nous ont été appris durant notre
parcours à Polytech dans le but de développer une application robuste et facilement extensible.

Au sein de ce rapport, nous présenterons notre démarche pour arriver aux objectifs fixés, le déroulement du projet ainsi que les résultats 
et ce dont on en a tiré.

\chapter{Conception et décisions}

\section{Objectifs et contraintes}



\section{Moteur}

\subsection{Structure générale}

\subsection{Entités}

\subsection{World}

\subsection{Colliders}

\subsection{Fourmis}

\subsubsection{Comportement d'une fourmi}

\subsubsection{Colonie}

\subsubsection{Ressources}

\subsubsection{Pheromones}

\subsubsection{Entité vivante qui se déplace}

\subsubsection{Machine à états}

\section{Application}

\subsection{Présentation de la structure}

\subsubsection{Elements d'interface}

\chapter{Réalisation}

\section{Choix des outils et technologies}

\section{Moteur}

\section{Application}

\section{Tests}

\chapter{Résultats et perspectives}

\section{Etat des lieux}

\section{Apports}

\section{Evolutions possibles}


%--------------------------------------------------------------------------------
\chapter*{Conclusion}
\addcontentsline{toc}{chapter}{\numberline{}Conclusion}
\markboth{Conclusion}{}
 ultricies ut accumsan magna interdum. Nullam ut malesuada urna. Donec ut sem est. Curabitur ut neque elit. Sed aliquam sodales libero ut rutrum. Duis eu massa quam, rutrum posuere sapien. Suspendisse turpis nulla, eleifend ut faucibus consequat, laoreet varius tortor. Vestibulum accumsan sagittis hendrerit. Nunc tristique ligula quis ligula faucibus adipiscing. Aenean interdum, odio at volutpat interdum, justo nunc tristique elit, quis elementum nisi ante id arcu. Mauris commodo posuere cursus. Ut nec massa odio. Maecenas at eros arcu, quis rhoncus magna. Sed pellentesque dictum nulla nec pretium.

%--------------------------------------------------------------------------------
%exemple de bibliographie
\begin{thebibliography}{99}
\label{sec:biblio}
\bibitem{ref1}  détail bibliographique de la ref1
\bibitem{ref2}  détail bibliographique de la ref2
\bibitem{ref3}  détail bibliographique de la ref3
\bibitem{ref4}  détail bibliographique de la ref4
\bibitem{ref5}  détail bibliographique de la ref5
\end{thebibliography}


%--------------------------------------------------------------------------------
%si on donne des annexes :
\appendix
\addcontentsline{toc}{part}{\numberline{}Annexes}

%--------------------------------------------------------------------------------

\chapter{Liens utiles\label{sec:liens_utiles}}
Voici une petite liste d'url intéressantes au sujet de ce projet :

\begin{itemize}
\item {\url{https://github.com/PolyNoradrenalin/AntSimulator} Répertoire GitHub du projet}
\end{itemize}

%--------------------------------------------------------------------------------
%index : attention, le fichier dindex .ind doit avoir le même nom que le fichier .tex
%\printindex

%--------------------------------------------------------------------------------
%page du dos de couverture :

\resume{Integer lorem purus, rutrum quis lacinia in, egestas ut urna. Donec elementum mi id nisi blandit quis ultricies risus semper. Nulla congue tincidunt diam, id tincidunt mauris euismod nec. Nullam faucibus dapibus eros, at consequat odio rutrum quis. Curabitur nisl sem, suscipit in mattis eu, varius a mauris. Ut a augue ac augue fringilla egestas. Etiam non augue felis, in convallis nisi. Maecenas id urna ut justo tempor laoreet in eu ligula. Duis non erat vitae eros rhoncus rutrum sit amet at lorem. Ut tempor cursus ligula, eu bibendum ligula adipiscing eu. Fusce feugiat aliquam dolor, nec interdum nisl convallis vitae.}

\motcles{???, ????, ?????, ?????????, ??, ????}

\abstract{Integer lorem purus, rutrum quis lacinia in, egestas ut urna. Donec elementum mi id nisi blandit quis ultricies risus semper. Nulla congue tincidunt diam, id tincidunt mauris euismod nec. Nullam faucibus dapibus eros, at consequat odio rutrum quis. Curabitur nisl sem, suscipit in mattis eu, varius a mauris. Ut a augue ac augue fringilla egestas. Etiam non augue felis, in convallis nisi. Maecenas id urna ut justo tempor laoreet in eu ligula. Duis non erat vitae eros rhoncus rutrum sit amet at lorem. Ut tempor cursus ligula, eu bibendum ligula adipiscing eu. Fusce feugiat aliquam dolor, nec interdum nisl convallis vitae.}

\keywords{???, ????, ?????, ?????????, ??, ????}


\makedernierepage


\end{document}
%%FIN du fichier